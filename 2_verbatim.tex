% !TEX TS-program = pdflatex -shell-escape
% !TEX encoding = UTF-8 Unicode
% !TEX spellcheck = en-US
% !TEX root = 0_article.tex
% Simplified BSD License (c) 2013, Andrew Pinkham

\subsection{\texttt{verbatim} Environment}

Here is an example that uses \LaTeX 's \verb|verbatim| environment.

\addlisting{Verbatim Example} % manually add code below to list of examples
\begin{verbatim}
from django.db import models

class Poll(models.Model):
    question = models.CharField(max_length=200)
    pub_date = models.DateTimeField('date published')

class Choice(models.Model):
    poll = models.ForeignKey(Poll)
    choice_text = models.CharField(max_length=200)
    votes = models.IntegerField(default=0)
\end{verbatim}

Note that the quotes used are standard programming quotes, as opposed to \LaTeX 's backtick-quote combination. This is possible because of the \verb|upquote| package, included in the preamble of the document.

\LaTeX{} further supplies the \verb|\verb| command, allowing for inline formatting.

On a related note, note that the \verb|\verb| command can only be used in captions if the entire command is preceded by the \verb|\cprotect| command, included in the \verb|cprotect| package, which you can read more about \link{http://www.ctan.org/tex-archive/macros/latex/contrib/cprotect}{on the documentation provided by CTAN}. An example is demonstrated in the caption of listing~\ref{code:django:models_basic_minted} on page~\pageref{code:django:models_basic_minted}.

Finally, text in chapter titles (or section, subsection, etc) are formatted with \verb|\texttt| in this document, as the use of \verb|\verb| in these areas appears to conflict with the \verb|hyperref| package. We could use \verb|\texttt| in captions as well, but that would defeat our purpose of demonstrating all the tools.
