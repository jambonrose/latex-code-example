% !TEX TS-program = pdflatex -shell-escape
% !TEX encoding = UTF-8 Unicode
% !TEX spellcheck = en-US
% !TEX root = 0_article.tex
% Simplified BSD License (c) 2013, Andrew Pinkham

\subsection{\texttt{listings} Package}

The \verb|listings| (plural) package provides the \verb|lstlisting| environment, which is a more sophisticated method for displaying source code. The environment will bold keywords, and comes with a slew of options, including the ability to add to the list of keywords. The environment also makes it easy to organize code listings, providing options for a caption and a label.

The example below opts to frame the entire example, displaying line numbers starting at 1, and declaring \verb|caption| and \verb|label| options.

\begin{lstlisting}[aboveskip=\baselineskip,%
                   basicstyle=\ttfamily,%
                   belowskip=\baselineskip,%
                   caption={[Listings Example]
                            \lstinline{models.py} from Django Tutorial using Listings},%
                   columns=fixed,%
                   firstnumber=1,%
                   frame=single,%
                   label=code:django:models_basic_listing,%
                   language=Python,%
                   numbers=left,%
                   showstringspaces=false,%
                   showspaces=false]
from django.db import models

class Poll(models.Model):
    question = models.CharField(max_length=200)
    pub_date = models.DateTimeField('date published')

class Choice(models.Model):
    poll = models.ForeignKey(Poll)
    choice_text = models.CharField(max_length=200)
    votes = models.IntegerField(default=0)
\end{lstlisting}

To avoid duplication of settings at each code listing, the \verb|listings| package provides a way of defining \verb|styles| in the preamble, using the \verb|\lstdefinestyle| command, which can then be invoked at the declaration of the \verb|lstlisting| environment.

To print a list of code listings, the \verb|listings| package provides the \verb|\lstlistoflistings| command. This document does not make use of this. Please see FILL IN LATER for information about the method used in this document.

For more on the \lstinline{listings} package, please see \link{https://en.wikibooks.org/wiki/LaTeX/Source_Code_Listings}{the wikibook on the subject}, or else \link{http://www.ctan.org/tex-archive/macros/latex/contrib/listings/}{the documentation provided by CTAN.}

\subsubsection{\texttt{lstinline} Command}

On top of an environment, the \lstinline{listings} package also provides a way to format inline text using \lstinline|\lstinline|, much like the \lstinline{\verb} command. Note that the settings for \lstinline|\lstinline| must be set in the options of \lstinline|\lstset| in the preamble.