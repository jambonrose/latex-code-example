% !TEX TS-program = pdflatex -shell-escape
% !TEX encoding = UTF-8 Unicode
% !TEX spellcheck = en-US
% !TEX root = 0_article.tex
% Simplified BSD License (c) 2013, Andrew Pinkham

\subsection{\texttt{listings} Package}

The following uses the \verb|listings| package to create an environment.

\noindent\minipage{\linewidth} % this avoids the code below being split across the page
\begin{lstlisting}[aboveskip=\baselineskip,%
                   basicstyle=\ttfamily,%
                   belowskip=\baselineskip,%
                   caption={\lstinline{models.py} from Django Tutorial using Listings},%
                   columns=fixed,%
                   firstnumber=1,%
                   frame=single,%
                   label=code:django:models_basic_listing,%
                   language=Python,%
                   numbers=left,%
                   showstringspaces=false,%
                   showspaces=false]
from django.db import models

class Poll(models.Model):
    question = models.CharField(max_length=200)
    pub_date = models.DateTimeField('date published')

class Choice(models.Model):
    poll = models.ForeignKey(Poll)
    choice_text = models.CharField(max_length=200)
    votes = models.IntegerField(default=0)
\end{lstlisting}
\endminipage % ends the samepage environment mentioned before the listing

On top of an environment, the \lstinline{listings} package also provides a way to format inline text using \lstinline|\lstinline{}|, replacing the \lstinline{verb} command. Note that the settings for \lstinline|\lstinline{}| must be set in the options of \lstinline|\lstset{}| in the preamble. For more on the \lstinline{listings} package, please see \link{https://en.wikibooks.org/wiki/LaTeX/Source_Code_Listings}{the wikibook on the subject}, or else \link{http://www.ctan.org/tex-archive/macros/latex/contrib/listings/}{the documentation provided by CTAN.}
Note that while not discussed here, it is possible to create a list of listings (a table of contents of all the listings).