% !TEX TS-program = pdflatex -shell-escape
% !TEX encoding = UTF-8 Unicode
% !TEX spellcheck = en-US

% Example: $ pdflatex -jobname=expected_output -shell-escape example.tex

% Simplified BSD License (c) 2013, Andrew Pinkham

% provide information for the document title
\title{Examples of Code Listings in \LaTeX{}}
\author{Andrew Pinkham} % use \and between authors if multiple
\date{\today}

% import preamble.tex
% allows us to use packages imported in file
% as well as any macro's we've created
% !TEX TS-program = pdflatex
% !TEX encoding = UTF-8 Unicode
% !TEX spellcheck = en-US
% !TEX root = example.tex

% Simplified BSD License (c) 2013, Andrew Pinkham

\documentclass[a4paper,12pt]{article}

%%%%%%%%%%%%%%%%%%%%%%%%%%%%%%%%%%%%%%%%%%%%%%%%%%%%%%%%%%%%%%%%%%%%%%%%%%%%%%%
% IMPORTS

% surprisingly, the order of some of these matter
% the following order works for me using TeXlive from MacTeX
% no guarantees for anyone/anything else

% font imports
\usepackage[T1]{fontenc}
\usepackage{courier}

% colors
\usepackage{color, xcolor}

% packages for appropriately floating captions
\usepackage{floatrow}
% the hypcap option below ensures that references/links
% will land the user at the top of the listing,
% regardless of caption position
\usepackage[hypcap]{caption}

% allow for frames/borders around text
% using TikZ allows for rounded corners
\usepackage[framemethod=TikZ]{mdframed}

% display 'lorem ipsum' automagically
\usepackage{lipsum}

% used for code listing
% minted (imported below) does not provide inline code - this does
\usepackage{listings}

% highlight code automagically via Pygments
% code listings will be organized numerically by section
% as opposed to chapter
\usepackage[section]{minted}

% single quotes in Verbatim environments are output as single quotes
% (silly, ain't it?)
\usepackage{upquote}
% the following hack fixes upquote use with minted
% when running Pygments 1.6
\AtBeginDocument{%
\def\PYZsq{\textquotesingle}%
}

% allow for Verbatim text in captions
\usepackage{cprotect}

% underline text correctly
% specify 'normalem' so that emphasis blocks are not replaced with underlined text
% used to create command to custom color hyperref links
\usepackage[normalem]{ulem}

% conditional logic in new environments
\usepackage{xstring}

% allows for the creation of columns
% demonstrated when printing the table of contents
\usepackage{multicol}

% safely print URLs and create links
\usepackage[hidelinks]{hyperref}

%%%%%%%%%%%%%%%%%%%%%%%%%%%%%%%%%%%%%%%%%%%%%%%%%%%%%%%%%%%%%%%%%%%%%%%%%%%%%%%
% DEFINITIONS
\lstset{%
    basicstyle=\ttfamily,%
    columns=fixed,%
    showstringspaces=false,%
    showspaces=false%
}

% new colors
\definecolor{light-gray}{gray}{0.95}
\definecolor{codeFrame}{gray}{0.3}
\definecolor{codeBG}{gray}{0.99}

%%%%%%%%%%%%%%%%%%%%%%%%%%%%%%%%%%%%%%%%%%%%%%%%%%%%%%%%%%%%%%%%%%%%%%%%%%%%%%%
% COMMANDS

\newcommand{\link}[2]{\href{#1}{\color{blue}\uline{#2}}}

%%%%%%%%%%%%%%%%%%%%%%%%%%%%%%%%%%%%%%%%%%%%%%%%%%%%%%%%%%%%%%%%%%%%%%%%%%%%%%%
% ENVRIONMENTS

% ASIDE BOX %%%%%%%%%%%%%%%%%%%%%%%%%%%%%%%%%%%%%%%%%%%%%%%%
\newmdenv[ %
		usetwoside=false,%
		nobreak=true,%
		roundcorner=7pt,%
		outerlinewidth=0.5pt,%
		linecolor=black,%
		skipabove=\topskip,%
		skipbelow=\topskip,%
		rightmargin=10pt,%
		leftmargin=10pt,%
		innertopmargin=\baselineskip,%
		innerbottommargin=\baselineskip,%
		innerrightmargin=10pt,%
		innerleftmargin=10pt,%  
		backgroundcolor=light-gray,%
		]{asidebox}

\newenvironment{aside}[1][Default]
    { %
      \begin{samepage} %
      \begin{asidebox}[frametitle=Aside - {#1}] %
    }
    {\end{asidebox}\end{samepage}}


% start the actual document
\begin{document}

% create a title using the title, author, and date info above
\maketitle

\begin{abstract}
This project is meant to help jump start writers looking to display source code in \LaTeX{} documents. The project provides a set of \TeX{} files and a PDF of the output of said \TeX{} files. The intention is that a reader may view the various code listing demonstrations in the PDF, and then read, learn, or simply copy the \LaTeX{} necessary to display their code in the same manner. Basic knowledge of \LaTeX{} is assumed.

We display code using the following: \verb|verbatim|, \verb|listings| (not to be confused with \verb|listing|), and \verb|minted|.

To avoid a monolithic \LaTeX{} document and make perusing the code simpler, each subsection is it's own \TeX{} file, as is the preamble.

Note that all code is licensed under the Simplified BSD License, a copy of which is included with the project code, found \link{https://github.com/jambonrose/latex-code-example}{on github}.
\end{abstract}


% below we print the table of contents and list of examples in two columns
\setlength{\columnseprule}{1pt} % add seperator between columns
\begin{multicols}{2} % create columns
\begin{footnotesize} % make fonts smaller
\renewcommand\contentsname{\small\centering Contents} % change title (font size)
\tableofcontents
\columnbreak
\renewcommand\listoflistingscaption{\small\centering List of Code Examples} % change title
\listoflistings
\end{footnotesize}
\end{multicols}

\newpage

\section[Basic Techniques]{Basic Tools and Techniques for Listing Code}

\subsection{\texttt{verbatim} Environment}

Here is an example that uses \LaTeX 's \verb|verbatim| environment.

\begin{verbatim}
from django.db import models

class Poll(models.Model):
    question = models.CharField(max_length=200)
    pub_date = models.DateTimeField('date published')

class Choice(models.Model):
    poll = models.ForeignKey(Poll)
    choice_text = models.CharField(max_length=200)
    votes = models.IntegerField(default=0)
\end{verbatim}

Note that the quotes used are standard programming quotes, as opposed to \LaTeX 's backtick-quote combination. This is possible because of the \verb|upquote| package, included in the preamble of the document.

\LaTeX{} further supplies the \verb|\verb| command, allowing for inline formatting.

On a related note, note that the \verb|\verb| command can only be used in captions if the entire command is preceded by the \verb|\cprotect| command, included in the \verb|cprotect| package, which you can read more about \link{http://www.ctan.org/tex-archive/macros/latex/contrib/cprotect}{on the documentation provided by CTAN}. An example is demonstrated in the caption of listing~\ref{code:django:models_basic_minted} on page~\pageref{code:django:models_basic_minted}.

Finally, text in chapter titles (or section, subsection, etc) are formatted with \verb|\texttt| in this document, as the use of \verb|\verb| in these areas appears to conflict with the \verb|hyperref| package. We could use \verb|\texttt| in captions as well, but that would defeat our purpose of demonstrating all the tools.

\subsection{\texttt{listings} Package}

The following uses the \verb|listings| package to create an environment.

\noindent\minipage{\linewidth} % this avoids the code below being split across the page
\begin{lstlisting}[aboveskip=\baselineskip,%
                   basicstyle=\ttfamily,%
                   belowskip=\baselineskip,%
                   caption={\lstinline{models.py} from Django Tutorial using Listings},%
                   columns=fixed,%
                   firstnumber=1,%
                   frame=single,%
                   label=code:django:models_basic_listing,%
                   language=Python,%
                   numbers=left,%
                   showstringspaces=false,%
                   showspaces=false]
from django.db import models

class Poll(models.Model):
    question = models.CharField(max_length=200)
    pub_date = models.DateTimeField('date published')

class Choice(models.Model):
    poll = models.ForeignKey(Poll)
    choice_text = models.CharField(max_length=200)
    votes = models.IntegerField(default=0)
\end{lstlisting}
\endminipage % ends the samepage environment mentioned before the listing

On top of an environment, the \lstinline{listings} package also provides a way to format inline text using \lstinline|\lstinline{}|, replacing the \lstinline{verb} command. Note that the settings for \lstinline|\lstinline{}| must be set in the options of \lstinline|\lstset{}| in the preamble. For more on the \lstinline{listings} package, please see \link{https://en.wikibooks.org/wiki/LaTeX/Source_Code_Listings}{the wikibook on the subject}, or else \link{http://www.ctan.org/tex-archive/macros/latex/contrib/listings/}{the documentation provided by CTAN.}
Note that while not discussed here, it is possible to create a list of listings (a table of contents of all the listings).

\subsection{\texttt{Minted} Package}

The last environment I'll demonstrate is the \verb|minted| package. It takes code in the environment, and runs it through Python's \verb|pygments| library. The resulting colored \LaTeX{} syntax is placed in a \verb|verbatim| environment.

\begin{minted}{python}
from django.db import models

class Poll(models.Model):
    question = models.CharField(max_length=200)
    pub_date = models.DateTimeField('date published')

class Choice(models.Model):
    poll = models.ForeignKey(Poll)
    choice_text = models.CharField(max_length=200)
    votes = models.IntegerField(default=0)
\end{minted}

While \verb|minted| does not provide an inline formatting command, it does provide a shortcut command \verb|mint| for short code snippets.

\mint{python}/fib = lambda n: n if n < 2 else fib(n-1) + fib(n-2)/

The \verb|minted| environment also provides the ability to be inserted into listings, with labels and captions.

\begin{listing}[H]
\begin{minted}{python}
from django.db import models

class Poll(models.Model):
    question = models.CharField(max_length=200)
    pub_date = models.DateTimeField('date published')

class Choice(models.Model):
    poll = models.ForeignKey(Poll)
    choice_text = models.CharField(max_length=200)
    votes = models.IntegerField(default=0)
\end{minted}
\cprotect\caption{\verb|models.py| from Django Tutorial using Minted}
\label{code:django:models_basic_minted}
\end{listing}

\section[Macros]{Custom Environments and Macros}

This section will demonstrate various macros defined in the preamble that you can use to make your document clearer.

\subsection{Aside}

While not for code, per se, this aside environment demonstrates the use of the \verb|mdframed| package.

\begin{aside}[Lorem Ipsum Example]
\lipsum[2] % alternative for more: \lipsum[1-3]
\end{aside}

\subsection{Python Code Macro}
Here is an example of the macro in the preamble for listing Python code.

\begin{pycode}%
      {\texttt{models.py} using Minted Macro}%
      {/models.py}[0]%
      {code:django:models_minted_macro}
from django.db import models

class Poll(models.Model):
    question = models.CharField(max_length=200)
    pub_date = models.DateTimeField('date published')

class Choice(models.Model):
    poll = models.ForeignKey(Poll)
    choice_text = models.CharField(max_length=200)
    votes = models.IntegerField(default=0)
\end{pycode}

\begin{center}
Simplified BSD License \textcopyright 2013, Andrew Pinkham
 \end{center}

\end{document}
